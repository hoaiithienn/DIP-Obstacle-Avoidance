\documentclass[10pt,conference,a4paper]{IEEEtran}
%\documentclass[12pt,draftcls,twocolumn]{IEEEtran}
\usepackage{ifpdf}
%\usepackage{cite}

% *** GRAPHICS RELATED PACKAGES ***
%
\ifCLASSINFOpdf
  \usepackage[pdftex]{graphicx}
  % declare the path(s) where your graphic files are
  \graphicspath{{../pdf/}{../jpeg/}}
  % and their extensions so you won't have to specify these with
  % every instance of \includegraphics
  \DeclareGraphicsExtensions{.pdf,.jpeg,.png}
\else
  % or other class option (dvipsone, dvipdf, if not using dvips). graphicx
  % will default to the driver specified in the system graphics.cfg if no
  % driver is specified.
  \usepackage[dvips]{graphicx}
  % declare the path(s) where your graphic files are
  \graphicspath{{../eps/}}
  % and their extensions so you won't have to specify these with
  % every instance of \includegraphics
  \DeclareGraphicsExtensions{.eps}
  \fi
%  \DeclareMathOperator{\vecOp}{\mathrm{vec}}
%	\DeclareMathOperator{\tr}{\mathrm{tr}}
%	\DeclareMathOperator{\E}{\mathbb{E}}
%	\DeclareMathOperator{\var}{\mathbb{V}\mathrm{ar}}
%	\DeclareMathOperator{\cov}{\mathrm{Cov}}
%	\DeclareMathOperator{\sgn}{\mathrm{sgn}}
%	\DeclareMathOperator{\etr}{\mathrm{etr}}
%	\DeclareMathOperator{\re}{\mathfrak{Re}}
%	\DeclareMathOperator{\im}{\mathfrak{Im}}
%\documentclass[10pt,conference,a4paper,compsocconf]{IEEEtran}
%\documentclass[11pt]{article}
%\usepackage{graphicx}
%\usepackage{amsmath,amssymb,hyperref,fancyhdr}%fancyhdr
\usepackage{amsmath,amsxtra,amssymb,amsthm,latexsym,amscd,amsfonts}
%\usepackage[utf8]{vietnam}
\usepackage[utf8]{vntex}
%\usepackage{fancyhdr}
%\pagestyle{fancy}
\renewcommand{\sectionmark}[1]{\markright{\MakeUppercase{#1}}{}}
%\renewcommand{\headrulewidth}{0pt}
\ifCLASSINFOpdf
\usepackage[pdftex]{graphicx}
\DeclareGraphicsExtensions{.pdf,.jpeg,.png,.jpg}
\else
\usepackage[dvips]{graphicx}
\DeclareGraphicsExtensions{.eps}
\fi
\usepackage{array}
\usepackage{epstopdf}
%\usepackage[font=footnotesize]{subfig}
%\hyphenation{op-tical net-works semi-conduc-tor}
\usepackage{anysize}
%\marginsize{2.4cm}{2.0cm}{3.8cm}{3.6cm}
% Use other margin values if the pdf file does not fit the word template (left: 2.4cm, right: 2.0cm, top: 3.8cm, bottom: 3.6cm)
\marginsize{2.4cm}{2.0cm}{3.2cm}{3.0cm}
%\marginsize{2.4cm}{1.4cm}{2.4cm}{2cm}
\usepackage{multicol}
\usepackage{balance}

\hyphenation{op-tical net-works semi-conduc-tor}
\newtheorem{theorem}{Theorem}
\newtheorem{lemma}[theorem]{Lemma}
%\newtheorem{proposition}[theorem]{Proposition}
\newtheorem{corollary}[theorem]{Corollary}
\newtheorem{proposition}{Bổ Đề}

\makeatletter
\def\ScaleIfNeeded{\ifdim\Gin@nat@width>\linewidth\linewidth\else\Gin@nat@width\fi}

\begin{document}
\columnsep=0.63cm
%---Mathematical symbols ------------------------------------------------------
\def\mathbi#1{\boldsymbol{#1}}
\def\erfc{\:\mathrm{erfc}}
\def\arg{\:\mathrm{arg}}
\def\E{\:\mathrm{E}}
\def\sinc{\:\mathrm{sinc}}
\def\T{\mathrm{T}}
\def\H{\mathrm{H}}
\newcommand{\bigsize}{\fontsize{16pt}{20pt}\selectfont}

%
% paper title
% can use linebreaks \\ within to get better formatting as desired
\include{Abbr}
\title{NHẬN DIỆN VẬT CẢN DI ĐỘNG TRONG VIDEO}

\author{
\IEEEauthorblockN{
Lương Hoài Thiện\IEEEauthorrefmark{1},
Hà Huy Dũng\IEEEauthorrefmark{1},
Nguyễn Huỳnh Đức\IEEEauthorrefmark{1},
Trần Lê Đức Trung\IEEEauthorrefmark{1},
} 
\IEEEauthorblockA{\IEEEauthorrefmark{1} Trường Đại học Bách Khoa TP. Hồ Chí Minh\\
		Email: \{1513202, 1510551, 1510797, 1513746\}@hcmut.edu.vn}
}
\maketitle

\begin{abstract}
%\boldmath 
%%% ---------------------------- ABSTRACT ----------------------------




%%% -------------------------- END ABSTRACT --------------------------
\end{abstract}

\begin{IEEEkeywords}
Nhận diện vật thể di động, xử lý ảnh, xe tự hành, mạng học sâu
\end{IEEEkeywords}
\IEEEpeerreviewmaketitle
%
\section{GIỚI THIỆU}
%%% ---------------------------- INTRODUCTION ----------------------------




%%% -------------------------- END INTRODUCTION --------------------------


%
\section{MÔ HÌNH HỆ THỐNG}
\label{Sec:MoHinhHeThong}
%%% ---------------------------- ABSTRACT ----------------------------




%%% -------------------------- END ABSTRACT --------------------------

%
\section{ĐÁNH GIÁ HIỆU NĂNG HỆ THỐNG}
\label{Sec:DanhGiaHieuNangHeThong}
%
%%% ---------------------------- ABSTRACT ----------------------------




%%% -------------------------- END ABSTRACT --------------------------
%
\section{KẾT QUẢ MÔ PHỎNG}
\label{Sec:KetQuaMoPhong}
%

%
\section{KẾT LUẬN}
\label{Sec:KetLuan}
%

\bibliographystyle{IEEEtran}
\balance
\bibliography{reference}
%
\end{document}


