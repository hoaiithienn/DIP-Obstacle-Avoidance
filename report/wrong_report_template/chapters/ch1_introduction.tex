\chapter {GIỚI THIỆU ĐỀ TÀI}

\renewcommand{\chaptermark}[1]{\markboth{ Ch\thechapter:\ #1}{}}

% \renewcommand{\section}{\@startsection {section}{1}{5ex}%
% 					   {-3.5ex \@plus -1ex \@minus -.2ex}%
% 					   {2.3ex \@plus.2ex}%
% 					   {\normalfont\Large\bfseries}}

% \renewcommand{\subsection}{\@startsection {subsection}{1}{10ex}%
% 					   {-3.5ex \@plus -1ex \@minus -.2ex}%
% 					   {2.3ex \@plus.2ex}%
% 					   {\normalfont\Large\bfseries}}

\chaptermark{Giới thiệu đề tài}
	
\section{Bài toán đặt ra và mục tiêu}
	Yêu cầu bài toán: Viết chương trình đọc các video từ camera hành trình của xe hơi, nhận dạng được các vật cản di động và tính toán để biết được xe có thể tránh được hay cần dừng lại.
	\begin{itemize}
	  \item Vì các vật thể di động trên động có thể rất đa dạng và là một bài toán lớn nên trong phạm vi đồ án sẽ chỉ xét các vật thể di động thông dụng trên đường sau: người, xe đạp, xe hơi, xe máy, xe tải.
	  \item Việc quyết định xe có thể tránh được hay dừng lại tùy thuộc rất nhiều vào tình huống thực tế nên trong phạm vi đồ án sẽ chỉ cố gắng đưa ra các quyết định đơn giản: rẽ trái, rẽ phải, đi thẳng và dừng để tránh dẫn tới việc va chạm.
	\end{itemize}

	Theo như yêu cầu của bài toán thì thông số cần tối thiểu là False Negative, tức là có nhưng không nhận dạng được.
	

\section{Các phướng hướng giải quyết bài toán}
	Về nhận dạng các vật thể, hiện nay có các hướng giải quyết chính như sau:

	\begin{enumerate}
	  \item Machine Learning
	  \begin{itemize}
	    \item Viola-Jones object detection framework based on Haar features
	    \item Scale-invariant feature transform
	    \item Histogram of oriented gradients 
	  \end{itemize}
	  \item Deep Learning
	  \begin{itemize}
	    \item Region Proposals (R-CNN, Fast R-CNN, Faster R-CNN)
	    \item Single Shot MultiBox Detector
	    \item You Only Look Once
	    \item Single-Shot Refinement Neural Network for Object Detection (RefineDet)
	  \end{itemize}
	\end{enumerate}

	Các phương hướng tiếp cận của Machine Learning rất phụ thuộc vào đặc tính hình của video và cần phải chỉnh sửa rất nhiều thông số mới có thể hoạt động tốt trên các video nhất định. Vì vậy, đồ án này sẽ sử dụng phương hướng tiếp cận là Deep Learning, các giải thuật này cho kết quả tốt hơn và lượng dữ liệu cho bài toán hiện này là rất lớn.
	\\
	Cụ thể hơn, giải thuật sẽ sử dụng là thuật toán YOLO và việc đưa ra các quyết định sẽ dựa trên các dữ liệu đầu ra của thuật toán này.

\newpage
