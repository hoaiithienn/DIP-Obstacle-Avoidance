\chapter*{LỜI NÓI ĐẦU}
\addcontentsline{toc}{chapter}{LỜI NÓI ĐẦU}

Trong bối cảnh Cách mạng Công nghiệp 4.0 đang diễn ra mạnh mẽ ở khắp mọi nơi trên mọi lĩnh vực khác nhau, khái niệm Y tế 4.0 đang được sử dụng ngày càng nhiều trong các ngành công nghiệp y sinh, dịch vụ y tế. Trong mô hình Y tế 4.0 này, bên cạnh sự phát triển theo xu hướng 4.0 của việc ứng dụng các thiết bị chẩn đoán điều trị và công cụ quản lý y tế kĩ thuật số, xu hướng hiện nay cho thấy có một sự dịch chuyển rõ nét tính chất dịch vụ y tế, từ chức năng điều trị lâm sàng sang chức năng cá thể tự chăm sóc quản lý sức khỏe và tự theo dõi thường xuyên. Xu hướng này được thể hiện rõ nét nhất trên thị trường các thiết bị y tế thông minh, nơi mà ngày càng nhiều sản phẩm đeo được (wearables) được tích hợp các cảm biến đo chức năng sinh học như nhịp tim, huyết áp, ECG,… ra mắt trên thị trường và thu hút được sự chú ý lớn của người tiêu dùng.


Là những sinh viên trong thời đại 4.0, việc tìm hiểu và nắm bắt xu thế đó ngay từ trên giảng đường Đại học là một trong những việc quan trọng để có thể tạo lợi thế cho bản thân và đáp ứng được nhu cầu của thị trường sau khi tốt nghiệp. Do đó, để có thể tìm hiểu về xu hướng các thiết bị giám sát y tế cầm tay, trong khoảng thời gian từ tháng 6 đến tháng 11 năm 2019, dưới sự hướng dẫn của thầy GS. TS Lê Tiến Thường, nhóm đã thực hiện thiết kế thiết bị thu thập dữ liệu giấc ngủ và đề xuất phương pháp phân tích và phát hiện triệu chứng ngưng thở khi ngủ dùng mạng thần kinh học sâu.

Dưới sự hỗ trợ về mặt kiến thức của thầy GS. TS Lê Tiến Thường, nhóm đã hoàn thành dự án đúng hạn và đạt được các mục tiêu đề ra. Nhưng do thời gian có hạn, cũng như kiến thức còn nhiều hạn chế, nhóm chắc chắn không tránh khỏi những thiếu sót. Mong rằng nhóm sẽ nhận được các góp ý quý báu từ Thầy Cô phản biện để có thể khắc phục và phát triển các ý tưởng cho giai đoạn tiếp theo.

Nhóm chân thành cảm ơn sự hỗ trợ nhiệt tình của Thầy Lê Tiến Thường trong suốt thời gian qua và đồng thời cảm ơn những người bạn cùng nhóm vì đã cùng nhau hoàn thành dự án này

\vspace{1cm}

\tabto{10cm} Ngày 25/10/2019

\vspace{2cm}

\tabto{9.5cm} Nhóm sinh viên thực hiện


\newpage